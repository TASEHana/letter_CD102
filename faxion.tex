%\section{Analysis of the Synthetic Axion Data}\label{sec:faxion}
After TASEH finished collecting the CD102 data on November 15, 2021, 
the synthetic axion signals were injected into the cavity and read out via the 
same transmission line and amplification chain. The procedure 
to generate axion-like signals is summarized in 
Ref.~\cite{TASEHInstrumentation}. 
Due to the uncertainties on the losses of signal transmission
 lines, the synthetic axion signals are not used to perform an absolute 
calibration of the search sensitivity. Instead, 
a test with synthetic axion signals could be used to verify the procedures of 
data acquisition and physics analysis. The 
SNR of the frequency bin with maximum power from the 
synthetic axion signals, at 4.708970~GHz, was set to $\approx 3.35$.%, 
%corresponding to a power of $\approx 6.03 \times 10^{-13}$~W in a 1-kHz 
%frequency bin.  
%
The same analysis procedure for the CD102 data is applied 
to the data with synthetic axion signals. 
%Figure~\ref{fig:faxionstep} presents the individual raw power spectra in 
%the 24 frequency scans. Before combining 
%the 24 spectra, the SNR of the maximum-power bin is measured to be 
%3.577. %the SNR is slightly higher than 3.35 due to a 
%5\% difference in the noise fluctuation between the measurements from 
%the calibration and the measurements taken right before injecting 
%axion-like signals. 
%After the combination of the spectra and the merging of five frequency 
%bins, the SNRs increase to 4.74 and 6.12, respectively. In addition to the 
%injected synthetic axion signal, a candidate at 4.708006~GHz is found after 
%merging the spectra. Since it is not possible to perform a rescan, 
%the real axion data from the two scans that had resonant frequencies close to 
%the candidate frequency are added so to mimic the rescan; the candidate is 
%found to be a statistical fluctuation.  
%Figure~\ref{fig:faxioncombinemerge} presents 
%the SNR after combining the spectra that share the same frequency bins and 
%after merging five neighboring bins, respectively; the 24 scans of the 
%synthetic axion data and the two 
%scans of the real axion data are included and processed together. 
The analysis results of the synthetic axion signals prove that a power 
excess of more than 5$\sigma$ can be found at the expected frequencies via 
the standard analysis procedure.  

   
