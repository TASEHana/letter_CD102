%\section{Results} \label{sec:results}

Figure~\ref{fig:glimit} shows the limits on the axion-two-photon coupling 
\gagg\ and the ratio of the limits on the dimensionless parameter \ggamma\ 
with respect to the KSVZ benchmark value ($\left|g_\text{KSVZ}\right|=0.97$).  
The blue error band indicates the systematic uncertainties as discussed in 
Sec.~\ref{sec:sys}. No limits are placed for the frequency ranges of 
4.710170 -- 4.710190~GHz and 4.747301 -- 4.747380~GHz, which correspond to 
the external signals during the collection of the CD102 data. The limits on 
\gagg\ range from \lolimit\GeVinv\ to \hilimit\GeVinv, with an average 
value of \avelimit\GeVinv; the lowest value comes from the frequency bins with 
additional eight times more data from the rescans, while the highest value 
comes from the frequency bins near the boundaries of the spectrum. 
Figure~\ref{fig:gaggall} displays the \gagg\ limits obtained by TASEH 
together with those from the previous searches. 
The results of TASEH exclude the models with the axion-two-photon
coupling $\gagg\gtrsim \avelimit\GeVinv$, a factor of ten above the benchmark
KSVZ model for the mass range $\mlo < \ma < \mhi \muevcc$ (corresponding to 
the frequency range of $\flo < f_a < \fhi$~GHz). 


The central results shown in Figs.~\ref{fig:glimit}--\ref{fig:gaggall} are 
obtained assuming an axion signal line shape that follows 
Eq.~\eqref{eq:simplesignal}. The analysis that merges bins without 
assuming a signal line shape %[$L_{k}=\left.1\middle/5\right.$ 
%in Eq.~\eqref{eq:merge_weight}] 
results in $\approx5.5$\% larger values on the 
\gagg\ limits. If a Gaussian signal line shape with an FWHM of 2.5~kHz,  
about half of the axion line width in Eq.~\eqref{eq:simplesignal}, is 
assumed instead, the limits will be $\approx3.8$\% smaller than the central results. 



\begin{figure*} [htbp]
  \centering
  \includegraphics[width=12.9cm]{figures/TASEHonly_limits.png}
%  \includegraphics[width=17.2cm]{figures/TASEHonly_limits.png}
%  \includegraphics[width=8.6cm]{figures/TASEHonly_limits.png}
  \caption{The limits on \gagg\ and the ratio of the limits on 
\ggamma\ relative to $\left|g_\text{KSVZ}\right|=0.97$ 
  (inset) for the frequency range of 
\flo--\fhi~GHz. The blue error band indicates the systematic 
  uncertainties as discussed in Sec.~\ref{sec:sys}. The yellow 
 band in the inset shows the allowed region of \ggamma\ vs. $m_a$ 
 from various QCD axion models, while the blue and red dashed lines are the 
values predicted by the KSVZ and DFSZ benchmark models, respectively}
  \label{fig:glimit}
\end{figure*}


\begin{figure*} [htbp]
  \centering
 \includegraphics[width=12.9cm]{figures/RealData_limit_allexp.png}
  \caption{The limits on the axion-two-photon coupling \gagg\ for the 
frequency ranges of 0--8~GHz, from the CD102 data of TASEH and previous 
searches performed by the ADMX, CAPP, and HAYSTAC Collaborations. The gray 
band indicates the allowed region of \gagg\ vs. $m_a$ from various QCD axion 
models while the blue and red dashed lines are the values predicted by the 
KSVZ and DFSZ benchmark models, respectively.}
  \label{fig:gaggall}
\end{figure*}


%\begin{figure} [htbp]
%  \centering
% \includegraphics[width=8.6cm]{figures/limitratio_weights.png}
%  \caption{The ratios of the limits on \gagg\ from the merging without 
% assuming a signal line shape (blue) and from the merging with a 
% Gaussian weight (orange), with respect to the central results.}
%  \label{fig:limitratio}
%\end{figure}
