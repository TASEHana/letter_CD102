% ****** Start of file apssamp.tex ******
%
%   This file is part of the APS files in the REVTeX 4.2 distribution.
%   Version 4.2a of REVTeX, December 2014
%
%   Copyright (c) 2014 The American Physical Society.
%
%   See the REVTeX 4 README file for restrictions and more information.
%
% TeX'ing this file requires that you have AMS-LaTeX 2.0 installed
% as well as the rest of the prerequisites for REVTeX 4.2
%
% See the REVTeX 4 README file
% It also requires running BibTeX. The commands are as follows:
%
%  1)  latex apssamp.tex
%  2)  bibtex apssamp
%  3)  latex apssamp.tex
%  4)  latex apssamp.tex
%
\documentclass[%
% reprint,prl, %% for final paper
%superscriptaddress,
%groupedaddress,
%unsortedaddress,
%runinaddress,
%frontmatterverbose, 
preprint, %% for single-column, double-spacing
%preprintnumbers,
%nofootinbib,
%nobibnotes,
%bibnotes,
 amsmath,amssymb,
 aps,
%pra,
%prb,
%rmp,
%prstab,
%prstper,
%floatfix,
]{revtex4-2}
\usepackage[utf8]{inputenc}
\usepackage{graphicx}% Include figure files
\usepackage{dcolumn}% Align table columns on decimal point
\usepackage{bm}% bold math
\usepackage{hyperref}% add hypertext capabilities
\usepackage[mathlines]{lineno}% Enable numbering of text and display math
\usepackage{amsmath} %for multiline in equation
%\usepackage{comment}

\linenumbers %for preprint
\relax % Commence numbering lines
%\usepackage[showframe,%Uncomment any one of the following lines to test 
%%scale=0.7, marginratio={1:1, 2:3}, ignoreall,% default settings
%%text={7in,10in},centering,
%%margin=1.5in,
%%total={6.5in,8.75in}, top=1.2in, left=0.9in, includefoot,
%%height=10in,a5paper,hmargin={3cm,0.8in},
%]{geometry}

%abbreviation
\newcommand{\gagg}{\ensuremath{\left|g_{a\gamma\gamma}\right|}}
\newcommand{\ggamma}{\ensuremath{\left|g_{\gamma}\right|}}
\newcommand{\bgagg}{\ensuremath{g_{a\gamma\gamma}}}
\newcommand{\bggamma}{\ensuremath{g_{\gamma}}}
\newcommand{\ma}{\ensuremath{m_a}}
%\def\muev{\ensuremath{\mu~\mathrm{eV}}
\newcommand{\tsys}{\ensuremath{T_\text{sys}}}
\newcommand{\ta}{\ensuremath{T_\text{a}}}
%\newcommand{\muevcc}{\ensuremath{\,\mu\text{e\hspace{-.08em}V\hspace{-0.16em}/\hspace{-0.08em}}c^2}}
\newcommand{\muevcc}{\ensuremath{\,\mu\text{e\hspace{-.08em}V}}}
\newcommand{\MeV}{\ensuremath{\,\text{Me\hspace{-.08em}V}}}
\newcommand{\GeV}{\ensuremath{\,\text{Ge\hspace{-.08em}V}}}
\newcommand{\GeVinv}{\ensuremath{\,\text{Ge\hspace{-.08em}V}^{-1}}}
\newcommand{\flo}{\ensuremath{4.707506}}
\newcommand{\fhi}{\ensuremath{4.798145}}
\newcommand{\mlo}{\ensuremath{19.47}}
\newcommand{\mhi}{\ensuremath{19.84}}
\newcommand{\avelimit}{\ensuremath{7.7\times 10^{-14}}} 
\newcommand{\lolimit}{\ensuremath{4.4\times 10^{-14}}} 
\newcommand{\hilimit}{\ensuremath{8.3\times 10^{-14}}} 
% exact number of the average limit 7.6965 x 10^-14 
%\newcommand{\noise}{\ensuremath{2.3}}
\newcommand{\noise}{\ensuremath{1.9 - 2.2}}


\begin{document}

\preprint{APS/123-QED}

\title{First Results from the Taiwan Axion Search Experiment with Haloscope in the \mlo--\mhi\muevcc\ Mass Range}% Force line breaks with \\
\thanks{A footnote to the article title}%

\author{Ann Author}
 \altaffiliation[Also at ]{Physics Department, XYZ University.}%Lines break automatically or can be forced with \\
\author{Second Author}%
 \email{Second.Author@institution.edu}
\affiliation{%
 Authors' institution and/or address\\
 This line break forced with \textbackslash\textbackslash
}%

\collaboration{TASEH Collaboration}%\noaffiliation


\date{\today}% It is always \today, today,
             %  but any date may be explicitly specified

\begin{abstract}

 This Letter reports on the first results from the 
Taiwan Axion Search Experiment with Haloscope, a search for axions 
using a microwave cavity at frequencies between \flo\ and \fhi~GHz. 
Apart from the external signals, no candidates with 
a significance more than 3.355 were found. The experiment excludes 
models with the axion-two-photon 
coupling $\gagg\gtrsim \avelimit\GeVinv$, a factor of ten above the benchmark 
KSVZ model for the mass range $\mlo < \ma < \mhi \muevcc$, reaching 
a sensitivity three orders of magnitude better than any existing limits. 
 It is also the first time that a haloscope-type experiment places 
constraints on the \gagg\ in this mass region.



%\begin{description}
%\item[Usage]
%Secondary publications and information retrieval purposes.
%\item[Structure]
%You may use the \texttt{description} environment to structure your abstract;
%use the optional argument of the \verb+\item+ command to give the category of each item. 
%\end{description}
\end{abstract}

%\keywords{Suggested keywords}%Use showkeys class option if keyword
                              %display desired
\maketitle

%\tableofcontents %for preprint
%%%%%%%%%%%%%%%%%%%%%%%%%%%%%%%%%%%%%%%%%%%%%%%%%%%%%%%%%%%%%%%%%%%%%%%%%%%%%%%
%\section{Introduction} \label{sec:intro}
The axion is a hypothetical particle predicted as a consequence of a  
solution to the strong CP problem~\cite{strongCPI,strongCPII,strongCPIII}.
%solution to the strong CP problem~\cite{strongCPI,strongCPII,strongCPIII}, 
%i.e. why the CP symmetry is conserved in the strong 
%interactions when there is an explicit CP-violating term in the QCD 
%Lagrangian. In other words, why is the electric dipole moment 
%of the neutron so tiny:  
%$\left|d_n\right| < 1.8 \times10^{-26}~e\cdot\mathrm{cm}$~\cite{EDM,PDG}? 
The solution proposed by Peccei and Quinn is to introduce a new global 
Peccei-Quinn U(1)$_\mathrm{PQ}$ symmetry that is spontaneously broken; the 
axion is the pseudo Nambu-Goldstone boson of 
U(1)$_\mathrm{PQ}$~\cite{strongCPI}. 
Axions are abundantly produced during the QCD phase transition in 
the early universe and may constitute the dark matter (DM). 
%Refs~\cite{axionDMI,axionDMII,axionDMIII} also suggest that axions form a 
%Bose-Einstein condensate; this property explains the occurrence of 
%caustic rings in galactic halos. 
%In the post-inflationary PQ symmetry breaking scenario, where the PQ symmetry
%is broken after inflation, current calculations suggest a mass range of 
%1–-100~\muevcc\ for axions so that the cosmic axion density does not exceed 
%the 
%observed cold DM density~\cite{QCDCalI,QCDCalII,QCDCalIII,QCDCalIV,QCDCalV,QCDCalVI,QCDCalVII,QCDCalVIII,QCDCalIX,QCDCalX,QCDCalXI,QCDCalXII,QCDCalXIII}. 
%Therefore, axions are compelling because they may explain at the same 
%time puzzles that are on scales different by more than thirty orders of 
%magnitude. 
%
%
%
Axions could be detected and studied via their two-photon interaction, the
so-called ``inverse Primakoff effect''. For QCD axions, i.e. the axions 
proposed to solve the strong CP problem, the axion-two-photon coupling 
constant \bgagg\ is related to the mass of the axion \ma: 
\begin{equation}
 \bgagg = \left(\frac{\bggamma\alpha}{\pi \Lambda^2}\right)\ma, 
\label{eq:grelation}
\end{equation}
where \bggamma\ is a dimensionless model-dependent parameter, $\alpha$ is the 
fine-structure constant, $\Lambda=78~\MeV$ is a scale parameter that can 
be derived from the mass and the decay constant of the pion, and the ratio of 
the up to down quark masses. 
The numerical values of \bggamma\ are -0.97 and 0.36 
in the Kim-Shifman-Vainshtein-Zakharov (KSVZ)~\cite{KSVZI,KSVZII} and 
the Dine-Fischler-Srednicki-Zhitnitsky (DFSZ)~\cite{DFSZI,DFSZII} benchmark 
models, respectively. 

The detectors with the best sensitivities to axions with a mass of 
$\approx \muevcc$, as first put forward by 
Sikivie~\cite{SikivieI,SikivieII},  
are haloscopes consisting of a microwave cavity immersed in a strong static 
magnetic field and operated at a cryogenic temperature. 
In the presence of an external magnetic field, the ambient oscillating axion 
field drives the cavity and they resonate when the frequencies of the 
electromagnetic 
modes in the cavity match the microwave frequency $f$, where $f$ is set by 
the total energy of the axion: $hf=E_a=\ma c^2 + \frac{1}{2}\ma v^2$; the 
signal power is further delivered  
to the readout probe followed by a low-noise linear amplifier. 
The axion mass is unknown, therefore, 
the cavity resonator must allow the possibility to be tuned through a range
of possible axion masses. The Axion Dark Matter eXperiment (ADMX), 
one of the flagship dark matter search experiments, had developed and 
improved the cavity design and readout electronics over the years. 
The results from the previous 
versions of ADMX and the Generation 2 ADMX (ADMX G2) excluded the KSVZ 
benchmark model within the mass range of %1.9--3.69~$\mu$eV$/c^2$. 
1.9--4.2\muevcc\ and the DFSZ benchmark model for the mass ranges 
of 2.66--3.31 and 3.9--4.1\muevcc, 
respectively~\cite{ADMXI,ADMXII,ADMXIII,ADMXIV,ADMXV,ADMXVI,ADMXVII}. 
%In addition to having sensitivity to the more weakly coupled DFSZ axions, 
One of the major goals of ADMX G2 is to search for higher-mass axions in the range 
of 4--40\muevcc\ (1--10~GHz), similarly for the axion experiments that 
were established during the last ten years.  
The Haloscope at Yale Sensitive to Axion Cold dark matter 
(HAYSTAC) had performed searches first for the mass range of 
23.15--24\muevcc\ and later at around 17\muevcc; they excluded axions 
with $\ggamma\geq 1.38 \ggamma^\text{KSVZ}$ for $m_a=16.96-17.12$ and 
17.14--17.28\muevcc, respectively~\cite{HAYSTACI}. The Center 
for Axion and Precision Physics Research (CAPP) constructed 
and ran simultaneously several experiments targeting at 
different frequencies; they have pushed the limits towards the 
KSVZ value within a narrow mass region of 
10.7126--10.7186\muevcc~\cite{CAPPI}.
The QUest for AXions-$a\gamma$ (QUAX-$a\gamma$) also pushed their limits 
close to the upper bound of the QCD axion-two-photon couplings for 
$m_a\approx43\muevcc$~\cite{QUAX}.   

This paper presents the first results of a search for axions for the mass 
range of \mlo--\mhi\muevcc, 
from the Taiwan Axion Search Experiment with Haloscope (TASEH). 


%%%%%%%%%%%%%%%%%%%%%%%%%%%%%%%%%%%%%%%%%%%%%%%%%%%%%%%%%%%%%%%%%%%
%\subsection{The expected axion signal power and signal line shape}
%\label{sec:introsignal}

The signal power extracted from a microwave cavity on resonance is given 
by:
\begin{equation}
P_s = \left(\bggamma^2\frac{\alpha^2\hbar^3c^3\rho_a}{\pi^2\Lambda^4}\right)\times
\left(\omega_c\frac{1}{\mu_0}B_0^2VC_{mnl}Q_L\frac{\beta}{1+\beta}\right),
\label{eq:ps}
\end{equation}
where $\rho_a=0.45~\GeV/\mathrm{cm}^3$ is the local dark-matter density. 
The second set of parentheses contains parameters related to the experimental 
setup: the angular resonant frequency of the cavity $\omega_c$, 
the vacuum permeability $\mu_0$, the nominal strength of the external magnetic 
field $B_0$, the volume of the cavity $V$, and the loaded quality factor of the 
cavity 
\(Q_L=Q_0/(1+\beta)\), where $Q_0$ is the unloaded, intrinsic quality factor 
of the cavity and $\beta$ is the coupling coefficient which determines the amount 
of coupling of the signal to the receiver. The form factor $C_{mnl}$ is the 
normalized overlap of the electric field 
$\vec{\bm{E}}$, for a particular cavity resonant mode, with the external magnetic 
field $\vec{\bm{B}}$:
\begin{equation}
  C_{mnl} = \frac{\left[\int\left( \vec{\bm{B}}\cdot\vec{\bm{E}}_{mnl}\right) d^3\bm{x}\right]^2}{B_0^2V\int E_{mnl}^2 d^3\bm{x}}.
\label{eq:formfactor} 
\end{equation} 
Here, the magnetic field $\vec{\bm{B}}$ points mostly along the axial 
direction ($z$-axis) of the cavity. 
The field strength has a small variation along the radial and axial directions and 
$B_0$ is the nominal magnetic field strength. 
For cylindrical cavities, the largest form factor is from the 
TM$_{010}$ mode. The expected signal power derived from the experimental 
parameters of TASEH (see Table~\ref{tab:tasehbenchmark}) 
is $P_s\simeq 1.5\times10^{-24}$~W for a KSVZ axion with a 
mass of 19.5\muevcc. 

In the direct dark matter search experiments, several assumptions are 
made in order to derive a signal line shape. 
The density and the velocity distributions of DM are related to each other 
through the gravitational potential. The DM in the galactic halo is assumed 
to be virialized. The DM halo density distribution is assumed 
to be spherically symmetric and close to be isothermal, which results in a 
velocity distribution similar to the Maxwell-Boltzmann distribution. The 
distribution of the measured signal frequency can be further derived from the 
velocity distribution after a change of variables and set 
\(hf_a = \ma c^2\). Previous experimental results typically adopt the 
following function for frequency $f\ge f_a$: 
\begin{equation}
\mathcal{F}(f) = \frac{2}{\sqrt{\pi}}\sqrt{f-f_a}\left(\frac{3}{\alpha}\right)^{3/2}
e^{\frac{-3\left(f-f_a\right)}{\alpha}}, 
\label{eq:simplesignal}
\end{equation}
where $\alpha\equiv  f_a \left<v^2\right>/c^2$. For a Maxwell-Boltzmann velocity 
distribution, the variance $\left<v^2\right>$ and the most probable velocity 
(speed) $v_p$ are related to each other:
$\left<v^2\right>=3v_p^2/2=$(270~km/s)$^2$, where $v_p=220$~km/s is the local 
circular velocity of DM in the galactic rest frame. 
Equation~\eqref{eq:simplesignal} 
is modified if one considers that the relative velocity of the DM halo with 
respect to the Earth is not the same as the DM velocity in the galactic rest 
frame~\cite{SignalLineShapeI}. The velocity distributions shall also be 
truncated so that the DM velocity is not larger than the escape velocity of 
the Milky Way~\cite{Lisanti:2016jxe}. 
Several N-body simulations~\cite{Diemand:2008in,Springel:2008cc} follow 
structure formation from the initial DM density perturbations to the largest 
halo today and take into account the merger history of the Milky Way, rather 
than assuming that the Milky Way is in a steady state; the simulated results 
 suggest velocity distributions with more high-speed particles relative 
to the Maxwellian case~\cite{Navarro:1995iw,Burkert:1995yz}. However, these 
numerical simulations contain only DM particles; an inclusion of baryons may 
enhance the halo's central density due to a condensation of gas towards the 
center of the halo via an adiabatic 
contraction~\cite{Blumenthal:1985qy,Gnedin:2004cx}, or may reduce 
the density due to the supernova outflows, etc~\cite{Mashchenko:2007jp,Governato:2009bg}. 


In order to compare the results of TASEH with those of the former experiments, 
the analysis presented in this paper assumes an axion 
signal line shape by including Eq.~\eqref{eq:simplesignal} in the weights 
when merging the measured power from multiple frequency bins 
(see Sec.~\ref{sec:merge}). A signal line width 
$\Delta f_a=\ma\left<v^2\right>/h\simeq$~5~kHz, which is much smaller than 
the TASEH cavity line width $f_a/Q_L\simeq$~250~kHz, is assumed and 
five frequency bins are merged to perform the final analysis. For a signal 
line shape as described in Eq.~\eqref{eq:simplesignal}, a 5-kHz bandwidth 
includes about 95\% of the distribution. 
Still given the caveats above and a lack of 
strong evidence for any particular choice of the velocity distribution, 
two different scenarios are considered and their results are 
presented for comparison: (i) without an assumption of signal line shape, and 
(ii) assuming a Gaussian signal line shape with a narrower full width at half 
maximum (FWHM), see Sec.~\ref{sec:results} for more details. 
%In addition, a signal line width 
%$\Delta f_a=\ma\left<v^2\right>/h\simeq$~5~kHz, which is much smaller than 
%the TASEH cavity line width $f_a/Q_L\simeq$~250~kHz, is assumed and 
%five frequency bins are merged to perform the final analysis. For a signal 
%line shape as described in Eq.~\eqref{eq:simplesignal}, a 5-kHz bandwidth 
%includes about 95\% of the distribution.


 
%\subsection{The expected noise and the signal-to-noise ratio}
%\label{sec:intronoise}
Several physics processes can contribute to the total noise and all of them 
can be seen as Johnson thermal noise at some effective temperature, or the 
so-called system noise temperature \tsys. The total noise power in a 
bandwidth $b$ is then:
\begin{equation}
  P_n = k_B\tsys b, 
\end{equation}
where $k_B$ is the Boltzmann constant. 
The system noise temperature \tsys\ has three major components: 
\begin{equation}
%  k_B\tsys = hf\left(\frac{1}{e^{\left.hf\right/k_{B}T_\mathrm{cavity}}-1} + \frac{1}{2} \right) + k_B\ta. 
  \tsys = T_\text{b} + T_\text{qn}  + \ta,
\label{eq:pn}
\end{equation}
where 
\begin{equation}
   T_\text{qn} = \frac{1}{2}\left. hf\middle/k_B\right..
\end{equation}
 The three terms in Eq.~\eqref{eq:pn} correspond to the effective 
temperatures of the following noise sources: (i) $T_\text{b}$, the blackbody 
radiation from the cavity at a physical temperature $T_\mathrm{c}$, (ii) 
$T_\text{qn}$, the quantum noise 
associated with the zero-point fluctuation of the vacuum, and (iii) \ta, the 
noise added by the receiver (mainly from the first-stage amplifier). 
Equation~\eqref{eq:pn} implies 
that the noise spectrum has little dependence on the frequency 
(white spectrum) for the narrow bandwidth considered in the experiment. 
However, apart from the flat baseline as described by 
Eq.~\eqref{eq:pn}, the noise spectrum observed by TASEH has an additional 
component with a Lorentzian shape due to the higher temperature at the 
cavity with respect to the temperature in the dilution refrigerator. 
More details may be found in Ref.~\cite{TASEHAnalysis}. 
The Lorentzian component will be removed from the measured spectrum and only 
the baseline \tsys\ will be used in the final analysis (Sec.~\ref{sec:ana}). 

Using the operation parameters of TASEH in Table~\ref{tab:tasehbenchmark} and 
the results from the calibration of readout electronics, 
the values of $T_\text{b}$, $T_\text{qn}$, and \ta\ are estimated to be about 
0.07~K, 0.12~K, and \noise~K, respectively.  
Therefore, the baseline value of \tsys\ for TASEH 
is about 2.1--2.4~K, which gives a noise power of approximately 
$\left(1.5-1.7\right)\times 10^{-19}$~W within the 5-kHz axion signal 
line-width, five 
orders of magnitude larger than the signal. Nevertheless, what matters in the 
analysis is the signal significance, or the so-called signal-to-noise ratio 
(SNR) using the standard terminology of axion experiments, i.e. the ratio of 
the signal power to the fluctuation in the averaged noise power spectrum 
$\sigma_n$. 

%Assuming that the amplitude distribution of the noise voltage within a 
%bandwidth is Gaussian, the $\sigma_n$ can be expressed below 
%according to Dicke's Radiometer Equation~\cite{Dicke}: 
According to Dicke's Radiometer Equation~\cite{Dicke}, the $\sigma_n$ 
is given by: 
\begin{eqnarray}
 \sigma_n  &=&  \frac{P_n}{\sqrt{N_\text{avg}}}, \nonumber \\
           &=&  \frac{P_n}{\sqrt{t\Delta f}}, \nonumber \\
           &=&  k_B\tsys\sqrt{\frac{\Delta f}{t}} 
 \label{eq:sigman}
\end{eqnarray}
where $N_\text{avg}$ is the number of noise power spectra used in the 
average; it is related to the amount of data integration time $t$ 
and the bandwidth over which a single measurement is made $\Delta f$.  
The SNR will therefore be: 
\begin{eqnarray}
   \text{SNR} & = & \frac{P_s}{\sigma_n}, \nonumber \\
              & = & \frac{P_s}{k_B\tsys}\sqrt{\frac{t}{\Delta f}},
 \label{eq:SNR}
\end{eqnarray}  
Combining Eq.~\eqref{eq:ps} and Eq.~\eqref{eq:SNR},
one could see that the SNR is maximized by an experimental setup with 
a strong magnetic field, a large cavity volume, an efficient cavity 
resonant mode, a receiver with low system noise temperature, and a 
long integration time. 












%\section{Experimental Setup}\label{sec:taseh} 
The detector of TASEH is located at the Department of Physics, National 
Central University, Taiwan and housed within a cryogen-free dilution 
refrigerator (DR) from BlueFors. An 8-Tesla superconducting solenoid with a 
bore diameter of 76~mm and a length of 240 mm is integrated with the DR. 

The data for the analysis presented in this paper were collected by TASEH 
from October 13, 2021 to November 15, 2021, and termed as the CD102 data, 
where CD stands for ``cool down''. 
During the data taking, the cavity sat in the center of the magnet bore 
and was connected via holders to the mixing flange of the DR at a 
temperature of $T_{\rm mx}\approx$27~mK. 
The temperature of the cavity stayed at $T_\text{c}\simeq155$~mK, higher 
with respect to the 
DR; it is believed that the cavity had an accidental thermal contact with the 
radiation shield in the DR. 
The cavity, made of oxygen-free high-conductivity (OFHC) copper, has an 
effective volume of 0.234~L and is a two-cell cylinder split along 
the axial direction ($z$-axis). 
The cylindrical cavity has an inner radius of 2.5~cm and a 
height of 12~cm.  In order to maintain a smooth surface, the cavity underwent 
the processes of annealing, polishing, and chemical cleaning. The resonant 
frequency of the TM$_{010}$ mode can be tuned over the range of 
%4.717--4.999~GHz via the rotation of an off-axis OFHC copper tuning rod, from 
4.667--4.959~GHz via the rotation of an off-axis OFHC copper tuning rod, from 
the position closer to the cavity wall to the position closer to the cavity 
center (i.e. when the vector from the rotation axis to the tuning rod is 
at an angle of $0^\circ$ to $180^\circ$, with respect to the vector from the 
cavity center to the rotation axis). 
The CD102 data cover the frequency range of \flo--\fhi~GHz. 
There were 839 resonant-frequency steps in total, with a frequency difference 
of $\Delta f_\text{s}=95-115$~kHz between the steps. The value of 
$\Delta f_\text{s}$ was kept within 10\% of 105~kHz rather than 
a fixed value, such that the rotation angle of the tuning rod did not need to 
be fine-tuned and the operation time could be minimized; a 10\% variation of 
the $\Delta f_\text{s}$ is found to have no impact on the \gagg\ limits. 
Each resonant-frequency step is denoted as a ``scan'' 
and the data integration time was about 32-42 minutes. The integration 
time was determined based on the target \gagg\ limits and the experimental 
parameters in Table~\ref{tab:tasehbenchmark}; the variation of the integration 
time aimed to remove the frequency-dependence in the \gagg\ limits caused by   
frequency dependence of the added noise \ta. The form factor $C_{010}$ as 
defined in Eq.~\eqref{eq:formfactor} varies from 0.64 to 0.69 over the 
full frequency range.  
The intrinsic, unloaded quality factor $Q_0$ at the cryogenic temperature 
($T_\mathrm{c}\simeq 155$~mK) is $\simeq 60000$ at the frequency of 
4.74~GHz.

An output probe, made of a 50-$\Omega$ semi-rigid coaxial cable that was 
soldered to an SMA (SubMiniature version A) connector, was inserted into the 
cavity and its depth was set for 
$\beta\simeq2$.  The signal from the output probe was directed to an 
impedance-matched amplification chain. The first-stage amplifier was 
a low noise high-electron-mobility transistor (HEMT) amplifier with an 
effective noise temperature of $\approx 2$~K, mounted on the 4K flange. 
The signal was further amplified at room temperature via a 
three-stage post-amplifier, and down-converted 
and demodulated to in-phase (I) and quadrature (Q) components and digitized 
by an analog-to-digital converter with a sampling rate of 2~MHz. 
%The frequency resolution of the spectra was 1~kHz.

A more detailed description of the TASEH detector, the operation of the 
data run, and the calibration of the gain and added noise temperature of the 
whole amplification chain can be found in Ref.~\cite{TASEHInstrumentation}. 
See Table~\ref{tab:tasehbenchmark} for the benchmark experimental parameters 
that can be used to estimate the sensitivity of TASEH. 

\begin{table}
\caption{The benchmark experimental parameters for estimating the sensitivity 
of TASEH. The definitions of the parameters can be found in Sec.~\ref{sec:intro}. 
More details regarding the determination and the measurements of 
some of the parameters may be found in Ref.~\cite{TASEHInstrumentation}.} 
\label{tab:tasehbenchmark}
\begin{center}
\begin{tabular}{cr}
\hline\hline
 $f_\mathrm{lo}$ & \flo~GHz\\
 $f_\mathrm{hi}$ & \fhi~GHz \\
% $N_\text{step}$ & 839 \\
 $N_\text{step}$ & 837 \\
 $\Delta f_\text{s}$ & 95 -- 115 kHz \\
% $f_\text{resolution}$ & 1 kHz \\
 $B_0$  & 8 Tesla \\
 $V$ & 0.234 L \\ %234255~mm$^3$ 
 % $C_{010}$ & 0.65 \\
 % $Q_0$ & 60000 \\
 % $\beta$ & 2 \\
 $C_{010}$ & 0.64 -- 0.69 \\
 $Q_0$ & 59000 -- 65000 \\
 $\beta$ & 1.9 -- 2.3 \\
 $T_{\rm mx}$ & 27--28~mK\\
 $T_\mathrm{c}$ & 155~mK \\
 \ta & \noise~K \\
 $\Delta f_a$ & 5 kHz \\
\hline\hline
\end{tabular}
\end{center}
\end{table}

%\section{Calibration} \label{sec:hemtcalibration}
%\label{sec:calibration}

%The first stage of the amplification chain is HEMT, whose added noise 
%dominates the noise from the readout electronics and could be 
%expressed in terms of an effective temperature \ta\ 
%(see Sec.~\ref{sec:intronoise}).
The noise is one of the most important parameters for the axion searches. 
Therefore, calibration for the amplification chain is a 
crucial part in the operation of TASEH. In order to perform a calibration, 
the HEMT was connected to a heat source (a 50-$\Omega$ resistor) instead of 
the cavity; 
various values of input currents were sent to the source to change its 
temperature monitored by a thermometer. The power from the source 
was delivered following the same transmission line as that in the axion 
data running. 
The output power is fitted to a first-order polynomial, as a function of 
the source temperature, to extract the gain and added noise for the 
amplification chain. More details of the 
procedure can be found in Ref.~\cite{TASEHInstrumentation}. 

The calibration was carried out before, during, and after the data taking, 
which showed that the performance of the system was stable over time. The 
average of the added noise \ta\ over 19 measurements has the lowest value of 
1.9~K at the frequency of 4.8~GHz and the highest value of 
2.2~K at 4.72~GHz, as presented in Fig.~\ref{fig:hemtcalvsf}. 
The error bars are the RMS of \ta\ and the largest RMS is used to calculate 
the systematic uncertainty for the limits on \gagg. The light blue points in 
Fig.~\ref{fig:hemtcalvsf} are the noise from the axion data estimated by 
removing the gain and subtracting the contribution from the cavity noise, 
assuming 
that the presence of a narrow signal in the data would have no effect on the 
estimation. A good agreement between the results from the calibration  
and the ones estimated from the axion data is shown. The biggest 
difference is 0.076~K in the frequency range during which the data were 
recorded after an earthquake. The source of the difference is not understood, 
therefore, the difference is quoted as a systematic uncertainty together 
with the RMS of the noise.

%\begin{figure} [htbp]
%  \centering
%  \includegraphics[width=8.6cm]{figures/Avg_Noise_vs_Freq_run1to19_211118.pdf}
%  \caption{The average added noise obtained from the calibration (pink points)
% and the noise estimated from the axion data (light blue points) as a 
%function of frequency. The error bars on the pink points are the RMS 
%of the \ta, as computed from the 19 measurements for each frequency 
%in the calibration. 
%The blue curve is obtained after performing a fit to 
%the pink points and is used to estimate the \ta\ at each resonant 
%frequency of the cavity.}
%  \label{fig:hemtcalvsf}
%\end{figure}


  


%\section{Analysis Procedure} \label{sec:ana}
% \begin{flushleft}
%    \subsection{Analysis overview}
The goal of TASEH is to find the axion signal hidden in the noise. In 
order to achieve this, the analysis procedure includes the following steps:
    \begin{enumerate}
        %\item Read raw data (I, Q) from tdms file and do Fast Fourier transform every 1 millisecond of spectrum, then average all over spectra in 1 second to get power spectrum.
        \item Perform fast Fourier transform (FFT) on the 
IQ time series data to obtain the frequency-domain power spectrum.
        \item Apply the Savitzky-Golay (SG) filter to remove the structure 
of the background in the frequency-domain power spectrum.
        \item Combine all the spectra from different frequency scans with 
the weighting algorithm.
        \item Merge bins in the combined spectrum to maximize the SNR. 
       \item Rescan the frequency regions with candidates and set limits on 
      the axion-two-photon coupling \gagg\ if no candidates were found.
    \end{enumerate}

    The analysis follows the procedure similar to that 
developed by the HAYSTAC experiment~\cite{HAYSTACII}. The important points  
and formulas for each step are highlighted below as a reminder 
for the convenience of readers. Note there are a few  
small differences between the HAYSTAC analysis and the one presented here. 
In this paper, the uncertainties are considered to be uncorrelated between 
different frequency bins while Ref.~\cite{HAYSTACII} takes into account 
the correlation. The frequency-domain spectra processed by each intermediate 
step are shown. The central results of the \gagg\ limits assume the signal 
line shape described by Eq.~\eqref{eq:simplesignal} as in 
Ref.~\cite{HAYSTACII}. In addition, the limits without an assumption of 
signal line shape and the limits assuming a 
Gaussian signal with a narrower FWHM are 
shown for comparison in Sec.~\ref{sec:results}.

% \end{flushleft}

%\subsection{Tdms}
%\subsection{Fast Fourier transform}
%\label{sec:FFT}
The in-phase $I(t)$ and quadrature $Q(t)$ components of the time-domain 
data were sampled and saved in the TDMS 
(Technical Data Management Streaming) files - a 
binary format developed by National Instruments.
%Fast Fourier Transform (FFT) was performed to convert the data into frequency-dependent and into unit of power by using the equation:
The FFT is performed to convert the data into 
frequency-domain power spectrum in which the measured power is calculated 
using the following equation:

\begin{equation}
\label{eq:4.1}
    \text{Power} = \frac{|\text{FFT}(I+i \cdot Q)|^{2}}{N \cdot 2R},
\end{equation}
where $N$ is the number of data points ($N  = 2000$ in the TASEH 
CD102 data), and $R$ is the input resistance of the signal analyzer 
(50~$\Omega$).
The FFT is done for every one-millisecond subspectrum data. The integration 
time for each frequency scan was about 32-42 minutes, which resulted 
in 1920000 to 2520000 subspectra; an average over these subspectra gives 
the averaged frequency-domain power spectrum for each scan. 
The frequency span in the spectrum from each resonant-frequency scan is 
1.6~MHz while the 
resolution is 1~kHz, giving 1600 frequency bins in each spectrum.  


%\subsection{Remove the structure of the background}
%Figure \ref{fig:sg_result} is the spectrum in each step, somehow they all have a similar structure, in order to remove this structure, we will use a Savitzky Golay filter in every second to smooth the data Figure \ref{fig:After_Sg}.

In the absence of the axion signal, the output data spectrum is simply the 
noise from the cavity and the amplification chain. If axions are present 
in the cavity, the signal will be buried in the noise because the 
signal power is very weak. Therefore, the structure of the raw averaged 
output power spectrum, as shown in the upper left panel of 
Fig.~\ref{fig:raw_sg_power}, is dominated 
by the noise of the system and an explanation for the structure can be found 
in Appendix~\ref{sec:cavitynoise}. The SG 
filter~\cite{SGFilter}, a digital filter that can smooth data without 
distorting the signal tendency, is applied to remove the structure of the  
background. The SG filter is performed on the averaged spectrum of each 
frequency scan by fitting adjacent points of successive sub-sets of data with 
an $n^\text{th}$-order polynomial. The result depends on two parameters: 
the number of 
data points used for fitting, the so-called window width, and the order of 
the polynomial. If the window is too wide, the filter will not remove small 
structures, and if it is too narrow, it may kill the signal. 
%The window and the order were first chosen during the data taking based on 
%the structure of data, by requiring the ratio of the raw data to the filter 
%output consistent with unity.  
The window and the order were first chosen during the data taking, by 
requiring the ratio of the raw data to the filter 
output consistent with unity.  
After the data taking, they were optimized by injecting an axion signal on 
top of 
the noise data and found that they were consistent with the original choice 
(see Sec.~\ref{sec:sys}). 

The raw averaged power spectrum is divided by the output of the SG filter, 
then unity is subtracted from the ratio to get the dimensionless 
normalized spectrum (lower left panel of Fig.~\ref{fig:raw_sg_power}). The value 
in each bin of the normalized spectrum is the deviation of the 
averaged measured power from the SG-filter output (can be considered 
as the averaged noise power) relative to the SG output. The symbol 
$\delta$ and term ``RDP'' are used to denote the relative deviation of power 
in the normalized spectrum and also in the spectra processed with rescaling, 
combining, and merging afterwards; the value can be zero, positive, or negative. 
In the absence of the axion signal, the RDPs in the normalized spectrum are 
samples drawn from a Gaussian 
distribution with a zero mean and a standard deviation of 
$\left.1\middle/\sqrt{N_\text{spectra}}\right.$, where $N_\text{spectra}$ is 
the number of subspectra used to compute the average (see Sec.~\ref{sec:FFT} 
and the right panel of Fig.~\ref{fig:raw_sg_power}). 
If the axion signal exists, there will be a significant excess above zero. 
 
%After filtering, the normalized spectrum (Fig. \cite{fig:}) was obtained by dividing the raw spectrum by the output of the SG filter subtract 1. Therefore, if the signal exists, the excess power will be above 0.
During the data taking, the resonant frequency of the cavity was  
adjusted by the tuning bar so to scan a large range of frequencies and to 
reduce the uncertainty of the averaged noise power at the overlapped region. 
Therefore, the 
spectra of all the scans need to be combined to create one big spectrum. 
%Before doing this, 
%the normalized spectrum from each scan is rescaled and the rescaled spectrum, 
%shown in 
%Fig.~\ref{fig:rescaled_power}, is computed with the following formula:
Before doing this, the normalized spectrum from each scan is rescaled and 
the rescaled spectrum is computed with the following formula:
\begin{equation}
  \label{eq:respower_eqn}
  \delta_{ij}^\text{res} = R_{ij}\delta_{ij}^\text{norm},
\end{equation}
and the standard deviation of each bin is:
\begin{equation}
  \label{eq:ressigma_eqn}
  \sigma_{ij}^\text{res} = R_{ij}\sigma_{i}^\text{norm},
\end{equation}
where 
 \begin{equation}
 R_{ij} = \frac{k_{B}\tsys \Delta f_\text{bin} }{P_{ij}^\text{KSVZ} h_{ij}}, 
 \label{eq:Rratio}
 \end{equation}
and 
 \begin{equation}
 h_{ij} = \frac{1}{1 + 4Q_{Li}^{2}(\left.f_{ij}\middle/f_{ci}\right.-1)^2}. 
 \label{eq:Lorentz}
 \end{equation}
The $\delta_{ij}^\text{norm}$ ($\delta_{ij}^\text{res}$) and 
$\sigma_{i}^\text{norm}$ ($\sigma_{ij}^\text{res}$) are the 
RDP and the standard deviation of the $j^\text{th}$ frequency bin in 
the normalized (rescaled) spectrum from the 
$i^\text{th}$ resonant-frequency scan. 
The value of $\sigma_{i}^\text{norm}$ is derived from the spread of the 
RDPs over the 1600 frequency bins for the $i^\text{th}$ scan. 
The factor $R_{ij}$ is the ratio of 
the system noise power to the expected signal power of the KSVZ axion 
$P_{ij}^\text{KSVZ}$, with the Lorentzian cavity response $h_{ij}$ 
taken into account. 
The system-noise temperature \tsys\ is calculated following Eq.~\eqref{eq:pn},
 where the frequency dependence of the added-noise temperature \ta\ is 
obtained from the fitting function in Fig.~\ref{fig:hemtcalvsf}. 
The $\Delta f_\text{bin}$ is the bin width of spectrum (1~kHz). 
The factor $h_{ij}$ describes the Lorentzian response of the cavity, 
which depends on the loaded quality factor $Q_{Li}$ and the 
difference between the frequency $f_{ij}$ in bin $j$ and the resonant 
frequency $f_{ci}$. 
%
If a signal appears in a certain frequency bin $j$, its expected power 
will vary depending on the bin position due to the cavity's 
Lorentzian response. The rescaling will take into account this effect. 
The procedure of the normalization and the rescaling also ensures that a 
KSVZ axion signal will have a rescaled RDP $\delta_{ij}^\text{res}$ 
that is approximately equal to unity, if the signal power is distributed 
in only one frequency bin. 

\begin{figure} [htbp]
  \centering
  \includegraphics[width=6.5cm]{figures/RawPower_SGPower_Ratio_vs_Freq_Step_0100.pdf}
  \includegraphics[width=6.5cm]{figures/Histogram_RawPower_SGPower_Ratio_Step_0100.pdf}
  \caption{Upper left panel: The raw averaged power spectrum (red points) and the 
output of the SG filter (blue curve) of one scan. Lower left panel: The normalized 
spectrum,  derived by taking the ratio of the raw spectrum to the SG filter 
and subtracting unity from the ratio.
Right plot: Histogram of the normalized spectrum (lower panel in left plot) with a Gaussian 
fit; there are 1600 entries in total (from the 1600 frequency bins). 
The fitted mean and standard deviation are shown to be consistent with the prediction 
when the axion signal is not present.}
  \label{fig:raw_sg_power}
\end{figure}

%\begin{figure} [htbp]
%  \centering
%  \includegraphics[width=8.6cm]{figures/RescaledPower_vs_Freq_Step_0100.pdf}
%  \caption{
%  The rescaled spectrum, obtained by multiplying the RDPs in the normalized 
%spectrum with the ratio of the system noise power to the expected signal 
%power of the KSVZ axion, 
%with the Lorentzian response of the cavity taken into account.}
%  \label{fig:rescaled_power}
%\end{figure}



%First we will choose a window and a order, then move the window and fit the data a with a polynomial with chosen order, it is a kind of generalization  moving average characterized, if the windows is too big, then it will not remove small structures, if it too small, it may kill the signal, you can see that choosing an appropriate window is important, a way to test if the windows are appropriate is to see the system temperature calculated by  $\mu$ and $\sigma$.

%\begin{equation}
%    \label{eq:ts_mu}
%    \mu = k_{B} \cdot T_{S} \cdot \Delta f 
%\end{equation}
%\begin{equation}
%    \label{eq:ts_sigma}
%    \sigma = \frac{k_{B} \cdot T_{S} \cdot \Delta f}{\sqrt{N}}
%\end{equation}

%If we treat the spectrum after the SG filter as a pure noise, we know that the system temperature of a white noise can be calculated by $\mu$ and $\sigma$ (Eq.\eqref{eq:ts_mu}) (Eq.\eqref{eq:ts_sigma}),
%where $k_{B}$ is the Boltzmann constant, $T_{S}$ is the system temperature, $\Delta f$ is the frequency resolution and N is the number of averaging. \\
%If the chosen window is appropriate, the system temperature estimated from $\mu$ and $\sigma$ should be consistent  with each other. \\

%We also did some studies to check if the SG filter removed the axion signal. Assuming a signal bandwidth of 5 kHz, we added the signal into noise spectrum and applied the SG filter. The result shows that the filter does not suppress the axion signal as given in Fig.\ref{fig:weighted_snr}.

%about whether the sg filter will remove the Axion signal, assuming the Axion signal bandwidth is 5KHz, add the signal in a white noise and apply the sg filter , the results show that it will not affect much before and after use, after divide the sg filter result, we will subtract 1 to make the value become 1.

%\begin{figure}[h]
%    \begin{minipage}[h]{.5\textwidth}
%    \centering
%    \includegraphics[width=0.8\textwidth, height = 0.5\textwidth]{Figure/sg_simulation.png}
%    \caption{The simulation for testing the effect of SG filter.}
%    \label{fig:weighted_snr}
%%    \end{minipage}%
%\end{figure}
%\begin{figure}[h]
%    \begin{minipage}[t]{.5\textwidth}
%    \centering
%    \includegraphics[width=0.8\textwidth,height = 0.5\textwidth]{Figure/weighted_snr.png}
%    \caption{Weighed SNR}
%    \label{fig:weighted_snr}
%%    \end{minipage}%
%\end{figure}

%\subsection{Combine the spectra with the weighting algorithm} 
%\label{sec:weighting_algorithm}

The purpose of the weighting algorithm is to add the spectra from different 
resonant-frequency scans,
 particularly for the frequency bins that appear in multiple spectra.  
Each spectrum was collected with a different cavity resonant frequency. 
Therefore, if a signal appears in a certain frequency bin $j$, due to the
 difference in the resonant frequency and the Lorentzian response, the 
expected signal
 power will be different in each spectrum $i$. The weighting algorithm is 
expected to take this into account with a weight calculated for each bin $j$ of
 the rescaled spectrum $i$, as defined below: %in Eq.~\eqref{eq:weight}:
\begin{equation}
    \label{eq:weight}
    %    {w_{n}}^{i} = \frac{h \cdot p}{(\sigma_{n}^{i})^{2}}
    {w_{ijn}} = \frac{\Gamma_{ijn}}{(\sigma_{ij}^\text{res})^{2}}.
\end{equation}
Note, the symbol $\Gamma_{ijn}=1$ if the $j^\text{th}$ frequency bin in the 
$i^\text{th}$ rescaled spectrum correspond to the same frequency in 
the $n^\text{th}$ bin of the combined spectrum; otherwise, $\Gamma_{ijn}=0$.

The RDP $\delta^\text{com}_{n}$ and the standard deviation 
$\sigma^\text{com}_{n}$ of the $n^\text{th}$ bin in the combined spectrum are 
calculated using Eq.~\eqref{eq:comb_power} and Eq.~\eqref{eq:comb_sigma}, 
respectively. The SNR$^\text{com}_{n}$ is the ratio of 
$\delta^\text{com}_{n}$ to 
$\sigma^\text{com}_{n}$ as given in Eq.~\eqref{eq:comb_snr}. 
%Figure~\ref{fig:power_sigma_comb} and Fig.~\ref{fig:SNR_comb} show the power, 
%the standard deviation, and the SNR of the combined spectrum, respectively.
%Figure~\ref{fig:power_sigma_comb} and Fig.~\ref{fig:SNR_comb} show the RDP, 
%the standard deviation, and the SNR of the combined spectrum, respectively.
Figure~\ref{fig:SNR_comb} shows the SNR of the combined spectrum. 

\begin{equation}
    \label{eq:comb_power}
%    \delta_{n}^\text{com} = \frac{ \sum_{1}^{k}\left(\delta_{ij}^\text{res} \cdot {w_{ij}}\right)}{\sum_{1}^{k} {w_{ij}}},
    \delta_{n}^\text{com} = \frac{ \sum\limits_{i}\sum\limits_{j}\left(\delta_{ij}^\text{res} \cdot {w_{ijn}}\right)}{\sum\limits_{i}\sum\limits_{j} {w_{ijn}}},
\end{equation}
\begin{equation}
    \label{eq:comb_sigma}
%    \sigma_{n}^\text{com} = \frac{ \sqrt{\sum_{1}^{k}(\sigma_{ij}^\text{res} \cdot {w_{ij}})^2}}{\sum_{1}^{k} {w_{ij}}},
    \sigma_{n}^\text{com} = \frac{ \sqrt{\sum\limits_{i}\sum\limits_{j}(\sigma_{ij}^\text{res} \cdot {w_{ijn}})^2}}{\sum\limits_{i}\sum\limits_{j} {w_{ijn}}},
\end{equation}
\begin{equation}
    \label{eq:comb_snr}
%    \text{SNR}_{n}^\text{com} = \frac{\delta^\text{com}_{n}}{\sigma^\text{com}_{n}}= \frac{\sum_{1}^{k}\left(\delta_{ij}^{res} \cdot {w_{ij}}\right)}{ \sqrt{\sum_{1}^{k}(\sigma_{ij}^{res} \cdot {w_{ij}})^2}},
    \text{SNR}_{n}^\text{com} = \frac{\delta^\text{com}_{n}}{\sigma^\text{com}_{n}}= \frac{\sum\limits_{i}\sum\limits_{j}\left(\delta_{ij}^{res} \cdot {w_{ijn}}\right)}{ \sqrt{\sum\limits_{i}\sum\limits_{j}(\sigma_{ij}^{res} \cdot {w_{ijn}})^2}}.
\end{equation} 
For each bin $n$ in the combined spectrum, there are $m_n$ non-vanishing 
contributions to the sums above. The value of $m_n$ ranges from 2 to 26; 
in general the leftmost bin or the bin with the smallest frequency 
(the rightmost bin or the bin with the highest frequency) in each scan has 
the minimum (maximum) number of $m_n$. 


%\begin{figure}[h]
%    \centering
%    \includegraphics[width=8.6cm]{figures/Power_CombSpectrum_AxionRun_AllSteps_Rescan_SG4_W201_LqWeight.png}
%    \includegraphics[width=8.6cm]{figures/Sigma_CombSpectrum_AxionRun_AllSteps_Rescan_SG4_W201_LqWeight.png}
%    \caption{The combined RDP $\delta$ following Eq.~\eqref{eq:comb_power} 
%(upper) and the standard deviation $\sigma$ derived from 
%Eq.~\eqref{eq:comb_sigma} (lower).}
%    \label{fig:power_sigma_comb}
%\end{figure}

\begin{figure}[hbt!]
    \centering
    \includegraphics[width=8.6cm]{figures/SNR_CombSpectrum_AxionRun_AllSteps_Rescan_SG4_W201_LqWeight.png}
    \caption{The signal-to-noise ratio (SNR) calculated using 
Eq.\eqref{eq:comb_snr} of the combined spectrum. }
    \label{fig:SNR_comb}
\end{figure}


%where ${\delta_n^i}$ and ${\sigma_n^i}$ are the measured power and the corresponding standard deviation of the ${n^{\mathrm{th}}}$ frequency bin of the ${i^{\mathrm{th}}}$ spectrum., From the weighted power in ${n^{th}}$ bin in Eq.\eqref{eq:weighted_power}, and weighted $\sigma$ in Eq.\eqref{eq:weighted_sigma}, we can get our Signal to noise ratio as Eq.\eqref{eq:weighted_SNR}

%\subsection{Merge bins}
%\label{sec:merge}

The expected axion bandwidth is about 5~kHz at the frequency of 
$\approx5$~GHz. 
In this paper, the interested frequency range is \flo -- \fhi~GHz and the bin 
width is 1~kHz. Therefore, in order to maximize the SNR, a running window of 
five consecutive bins in the combined spectrum is applied and the five bins 
within each window are merged to construct a final spectrum.  
The purpose of using a running window is to avoid the signal power broken 
into different neighboring bins of the merged spectrum. 
The number of bins for merging is studied by injecting 
simulated axion signals on top of the CD102 data and optimized based 
on the SNR. 
%For a signal line shape as described in Eq.~\eqref{eq:simplesignal}, 
%five frequency bins include about 95\% of the distribution.  
%Before defining the 
%weights for merging, 
%the power and the standard deviation of each bin in the combined spectrum are 
%multiplied with $M=5$: $\delta^{c}_n \rightarrow M\delta^\text{com}_n$ and 
%$\sigma^{c}_n \rightarrow M \sigma^\text{com}_n$. This rescaling gives the 
%expected mean of the normalized power $\mu^\text{com}_k = 1$ if a KSVZ axion 
%signal power leaves a fraction 1/$M$ of its power in the $k^\text{th}$ 
%bin of the combined spectrum.
Due to the nonuniform distribution of the axion signal 
[Eq.~\eqref{eq:simplesignal}],
the contributing bins need to be rescaled to have the same RDP, of which the 
standard deviation is used to define the maximum likelihood (ML)
weight for merging. The rescaling is performed by dividing the 
$\delta^\text{com}_{g+k-1}$ and $\sigma^\text{com}_{g+k-1}$ in the combined 
spectrum with an integral of the signal line shape $L_{k}$:

%[Eq.~\eqref{eq:Lq_integral}]:
\begin{equation}
  \label{eq:Lq_integral}
  L_{k} = \int_{f_a +\delta f_m + (k-1)\Delta f_\text{bin}}^{f_a +\delta f_m + k\Delta f_\text{bin}} \mathcal{F}(f) \,df,
\end{equation}
where the variable $k$ is the index within the group of bins for 
merging, the frequency $f_a=\left.\ma c^2\middle/h\right.$ is the axion
frequency, and $\delta f_m$ is the misalignment between $f_a$ and the lower
boundary of the $g^\text{th}$ bin in the merged spectrum.
The function $\mathcal{F}(f)$ has been defined in Eq.~\eqref{eq:simplesignal}.
In order to get a misalignment-independent line shape, instead of using an
$L_{k}$ that depends on $\delta f_m$, the average ($\bar{L}_{k}$) of
$L_{k}$ over the range of $\delta f_m$ is used.
In the analysis presented here, 
$\bar{L}_{k} = 0.23, 0.33, 0.21, 0.11, 0.06$ for $k = 1, ... 5$, respectively.
The misalignment effect as mentioned in the HAYSTAC paper~\cite{HAYSTACII}
has been studied and the results of the \gagg\ limits are found to be
insensitive to this effect.

The rescaled RDP ($\delta^\text{rs}_{g+k-1}$) and
standard deviation ($\sigma^\text{rs}_{g+k-1}$) are calculated:
\begin{equation}
  \label{eq:rescaled_delta_sigma_com}
  \begin{split}
  \delta^\text{rs}_{g+k-1} = \frac{\delta^\text{com}_{g+k-1}}{\bar{L}_{k}},\\
  \sigma^\text{rs}_{g+k-1} = \frac{\sigma^\text{com}_{g+k-1}}{\bar{L}_{k}}.
  \end{split}
\end{equation}
%Here, the $\delta^\text{rs}_{g+k-1}$ and $\sigma^\text{rs}_{g+k-1}$ are the 
%rescaled RDP and standard deviation that will be used later for merging.
The variable $g = 1,..,N-M+1$ is the index for the frequency bins in the final 
spectrum and $M = 5$ is the number of merged bin in 
this analysis. The numbers $N$ and $N-M+1$ are the total numbers of bins in 
the combined and final spectrum, respectively. After this rescaling 
procedure, a KSVZ axion signal is expected to have an RDP equal to unity for 
each bin of the five merged bins.    
%The integral $L_{k}$ is defined as: 
%\begin{equation}
%  \label{eq:Lq_integral}
%  L_{k} = \int_{f_a +\delta f_m + (k-1)\Delta f_\text{bin}}^{f_a +\delta f_m + k\Delta f_\text{bin}} \mathcal{F}(f) \,df,
%\end{equation}
%where the frequency $f_a=\left.\ma c^2\middle/h\right.$ is the axion 
%frequency, $\delta f_m$ is the misalignment between $f_a$ and the lower
%boundary of the $g^\text{th}$ bin in the merged spectrum.
%The function $\mathcal{F}(f)$ has been defined in Eq.~\eqref{eq:simplesignal}. 
%In order to get a misalignment-independent line shape, instead of using an 
%$L_{k}$ that depends on $\delta f_m$, the average ($\bar{L}_{k}$) of
%$L_{k}$ over the range of $\delta f_m$ is used.
%In the analysis presented here, 
%$\bar{L}_{k} = 0.23, 0.33, 0.21, 0.11, 0.06$ for $k = 1, ... 5$, respectively. 
%The misalignment effect as mentioned in the HAYSTAC paper~\cite{HAYSTACII} 
%has been studied and the results of the \gagg\ limits are found to be 
%insensitive to this effect. 

%The rescaled $\delta^\text{com}_{g+k-1}$ and $\sigma^\text{com}_{g+k-1}$ and 
%ML weight are defined:


%In order to get misalignment-independence line shape, the average ($L\bar_{k}$) of
%$L_{k}(\delta f_m)$ over the range $\delta f_m$ is used.
%The range of $\delta f_m$ is defined $-z \Delta f_\text{bin} < \delta f_m < (1-z) \Delta f_\text{bin}$,
%with $0 < z < 1$. 

%Then the maximum likelihood weights, defined in Eq.~\eqref{eq:merge_weight} 
%based on the Maxwellian line shape for axions [Eq.~\eqref{eq:simplesignal}], 
%are used to build the merged spectrum: 
And the ML weight is defined as: 
\begin{equation}
    \label{eq:merge_weight}
    w_{gk} = \frac{1}{(\sigma_{g+k-1}^\text{rs})^{2}} = \frac{\bar{L}_{k}^{2}}{(\sigma_{g+k-1}^\text{com})^{2}},
\end{equation}

% in which 
%\begin{equation}
%    \label{eq:Lq_integtal}
%    L_{k} = \int_{f_a +\delta f_m + (k-1)\Delta f_\text{bin}}^{f_a +\delta f_m + k\Delta f_\text{bin}} \mathcal{F}(f) \,df. 
%\end{equation}
%Here, $g = 1,..,N-M+1$ is the index for 
%the frequency bins in the final spectrum and $M=5$ is the number of merged 
%bins. 
%The total number of bins in the combined (final merged) spectrum is
%$N$ ($N-M+1$). 
%The variable $k$ is the index within the group of bins for merging and 
%$k = 1,..,M$. 
%The frequency $f_a=\left.\ma c^2\middle/h\right.$ is 
%the axion frequency, $\delta f_m$ is the misalignment between $f_a$ and the lower 
%boundary of the $g^\text{th}$ bin in the combined spectrum. 

%where $L_q$ is the integral of the line shape from the lower edge to higher 
%edge of ${q^{th}}$ bin. 

%We have a weight Eq.\eqref{eq:merge_weight}, where Lq is the area in ${q^{th}}$ bin and $\sigma_{q}$ is the weighted $\sigma$ in the $q^{\mathrm{th}}$ bin , Eq.\eqref{eq:axion_line_shape} is the axion CDM(cold dark matter) line shape, where $\big \langle \beta^{2} \big \rangle = \big \langle v^{2} \big \rangle /c^{2}$ and $\big \langle v^{2} \big \rangle = (270km/s)^{2}\ , \big \langle v^{2} \big \rangle$ is the squared virial velocity. Eq.\eqref{eq:L_q_integtal},

The RDP, the standard deviation, and the SNR of the merged spectrum are:

\begin{equation}
    \delta_{g}^\text{merged} = \frac{ \sum\limits_{k = 1}^{M}\left(\delta_{g+k-1}^\text{rs} \cdot {w_{gk}}\right)}{\sum\limits_{k = 1}^{M} {w_{gk}}} = \frac{\sum\limits_{k = 1}^{M}\frac{\delta_{g+k-1}^\text{com}}{\bar{L}_{k}} \cdot \left(\frac{\bar{L}_{k}}{\sigma_{g+k-1}^\text{com}}\right)^2} {\sum\limits_{k = 1}^{M}\left(\frac{\bar{L}_{k}}{\sigma_{g+k-1}^\text{com}}\right)^2},
    \label{eq:merged_power}
\end{equation}

\begin{eqnarray}
  \sigma_{g}^\text{merged} & =  & \frac{ \sqrt{\sum\limits_{k = 1}^{M} \left(\sigma_{g+k-1}^\text{rs} \cdot {w_{gk}}\right)^2}}{\sum\limits_{k = 1}^{M} {w_{gk}}} = \frac{\sqrt{\sum\limits_{k = 1}^{M} \left(\frac{\bar{L}_{k}}{\sigma_{g+k-1}^\text{com}}\right)^2}}{\sum\limits_{k = 1}^{M} \left(\frac{\bar{L}_{k}}{\sigma_{g+k-1}^\text{com}}\right)^2}  \nonumber \\
    & = & \frac{1}{\sqrt{\sum\limits_{k = 1}^{M} \left(\frac{\bar{L}_{k}}{\sigma_{g+k-1}^\text{com}}\right)^2}}
    \label{eq:merged_sigma}
\end{eqnarray}

\begin{equation}
    \label{eq:merged_snr}
    %    \text{SNR}_{g}^\text{merged} = \frac{\delta^\text{merged}_{g}}{\sigma^\text{merged}_{g}} = \frac{\sum\limits_{k = 1}^{M}\left(\delta_{g+k-1}^\text{com} \cdot {w_{gk}}\right)}{ \sqrt{\sum\limits_{k = 1}^{M} \left(\sigma_{g+k-1}^\text{com} \cdot {w_{gk}}\right)^2}},
    \text{SNR}_{g}^\text{merged} = \frac{\delta^\text{merged}_{g}}{\sigma^\text{merged}_{g}} = \frac{\sum\limits_{k = 1}^{M}\frac{\delta_{g+k-1}^\text{com}}{\bar{L}_{k}} \cdot \left(\frac{\bar{L}_{k}}{\sigma_{g+k-1}^\text{com}}\right)^2}{\sqrt{\sum\limits_{k = 1}^{M} \left(\frac{\bar{L}_{k}}{\sigma_{g+k-1}^\text{com}}\right)^2}}
\end{equation}

%For the flat distribution, $L_{k}$ is normalized by the number of merged bins 
%to have the expected power excess of 1 for KSVZ signal.

%where $M=5$ is the number of merged bins. $g = 1,..,N-M+1$ is the index for 
%the frequency bins in the final spectrum.  
%The total number of bins in the combined (final merged) spectrum is 
%$N$ ($N-M+1$). 
%Without the multiplication factor $M$ in 
%Eqs.~\eqref{eq:merged_power}--~\eqref{eq:merged_sigma}, the results of the two equations 
%simply provide the weighted averages of RDP and standard deviation, respectively.   
%Assuming that the axion signal leaves a fraction $1/M$ of its power in one 
%frequency bin, adding the factor $M$ converts the average into a summation of the values in the 
%five bins and can recover the full size of the axion signal. 

%Adding adjacent bin with a weight Eq.\eqref{eq:merged_power} Eq.\eqref{eq:merged_sigma}, where k is the number of adjacent bin to be merged, q is ${q^{th}}$ merged bin, with Eq.\eqref{eq:merged_sigma}, we can get our weighted merged power spectrum FIG.\ref{fig:merged_data}.


%\subsection{Rescan and set limits on \gagg} 
Before the collection of the CD102 data, a 5$\sigma$ SNR target was chosen, 
which corresponds to a candidate threshold of 3.355$\sigma$ at 95\% 
confidence level (C.L.).
 After the merging as described in Sec.~\ref{sec:merge}, if there were 
any potential signal with an SNR larger than 
3.355, a rescan would be proceeded to check if it were a real signal 
or a statistical fluctuation. 
The procedure of the CD102 data taking was to perform a rescan after 
covering every 10~MHz; the rescan was done by adjusting the tuning rod of the 
cavity so to match the resonant frequency to the frequency of the candidate. 
In total, 22 candidates with an SNR greater than 3.355 were found. 
Among them, 17 candidates were from the fluctuations because they were gone 
after a few rescans. 
The remaining five candidates, in the frequency ranges of 
4.710170 -- 4.710190~GHz and 4.747301 -- 4.747380~GHz, reached an SNR greater than 
4 after rescanning. The signals in the second frequency 
range were detected via a portable antenna outside the DR and found 
to come from the instruments in the laboratory, while the signals 
in the first frequency range were weaker but still present after 
turning off the external magnetic field. 
Therefore, these five candidates are considered external signals and 
no limits are placed for the above two frequency ranges.  
More details can be found in the 
TASEH instrumentation paper~\cite{TASEHInstrumentation}. 
%Figure~\ref{fig:power_sigma_merged} and Fig.~\ref{fig:SNR_merged} show the 
%RDP, the standard deviation, and the SNR of the merged spectrum after 
%including data from both the original scans and the rescans, respectively. 
Figure~\ref{fig:SNR_merged} shows the SNR of the merged spectrum after 
including data from both the original scans and the rescans. 

Since no candidates were found after the rescan, an upper limit on 
the signal power $P_s$ is derived by setting $P_s$ equal to 
$5\sigma_{g}^\text{merged}\times P_{g}^\text{KSVZ}$, where 
the $\sigma_{g}^\text{merged}$ and $P_{g}^\text{KSVZ}$ are 
the standard deviation and the expected signal power for the KSVZ axion 
for a certain frequency bin $g$ in the merged spectrum. 
Then, the 95\% C.L. limits on the dimensionless parameter 
\ggamma\ and the axion-two-photon coupling \gagg\ could be derived 
according to Eq.~\eqref{eq:ps} and Eq.~\eqref{eq:grelation}. 
See Sec.~\ref{sec:results} for the final limits including the systematic 
uncertainties.

%\begin{figure}[h]
%    \centering
%    \includegraphics[width=8.6cm]{figures/Power_GrandSpectrum_AxionRun_AllSteps_Rescan_Merged_5bin_SG4_W201_LqWeight.png}
%    \includegraphics[width=8.6cm]{figures/Sigma_GrandSpectrum_AxionRun_AllSteps_Rescan_Merged_5bin_SG4_W201_LqWeight.png}
%    \caption{The merged RDP $\delta$ following Eq.~\eqref{eq:merged_power} 
%(upper) and the standard deviation $\sigma$ derived from Eq.~\eqref{eq:merged_sigma} (lower). The results shown are obtained using data from both the 
%original scans and the rescans.}
%    \label{fig:power_sigma_merged}
%\end{figure}

\begin{figure}[hbt!]
    \centering
    \includegraphics[width=8.6cm]{figures/SNR_GrandSpectrum_AxionRun_AllSteps_Rescan_Merged_5bin_SG4_W201_LqWeight.png}
    \caption{The signal-to-noise ratio (SNR) calculated using Eq.~\eqref{eq:merged_snr} for the merged spectrum including data from both the original 
scans and the rescans. No candidate exceeds the threshold of 
$3.355\sigma$ (solid-black horizontal line). }
    \label{fig:SNR_merged}
\end{figure}

%\section{Analysis of the Synthetic Axion Data}\label{sec:faxion}
After TASEH finished collecting the CD102 data on November 15, 2021, 
the synthetic axion signals were injected into the cavity and read out via the 
same transmission line and amplification chain. The procedure 
to generate axion-like signals is summarized in 
Ref.~\cite{TASEHInstrumentation}. 
Due to the uncertainties on the losses of signal transmission
 lines, the synthetic axion signals are not used to perform an absolute 
calibration of the search sensitivity. Instead, 
a test with synthetic axion signals could be used to verify the procedures of 
data acquisition and physics analysis. The 
SNR of the frequency bin with maximum power from the 
synthetic axion signals, at 4.708970~GHz, was set to $\approx 3.35$.%, 
%corresponding to a power of $\approx 6.03 \times 10^{-13}$~W in a 1-kHz 
%frequency bin.  
%
The same analysis procedure for the CD102 data is applied 
to the data with synthetic axion signals. 
%Figure~\ref{fig:faxionstep} presents the individual raw power spectra in 
%the 24 frequency scans. Before combining 
%the 24 spectra, the SNR of the maximum-power bin is measured to be 
%3.577. %the SNR is slightly higher than 3.35 due to a 
%5\% difference in the noise fluctuation between the measurements from 
%the calibration and the measurements taken right before injecting 
%axion-like signals. 
%After the combination of the spectra and the merging of five frequency 
%bins, the SNRs increase to 4.74 and 6.12, respectively. In addition to the 
%injected synthetic axion signal, a candidate at 4.708006~GHz is found after 
%merging the spectra. Since it is not possible to perform a rescan, 
%the real axion data from the two scans that had resonant frequencies close to 
%the candidate frequency are added so to mimic the rescan; the candidate is 
%found to be a statistical fluctuation.  
%Figure~\ref{fig:faxioncombinemerge} presents 
%the SNR after combining the spectra that share the same frequency bins and 
%after merging five neighboring bins, respectively; the 24 scans of the 
%synthetic axion data and the two 
%scans of the real axion data are included and processed together. 
The analysis results of the synthetic axion signals prove that a power 
excess of more than 5$\sigma$ can be found at the expected frequencies via 
the standard analysis procedure.  

   

\section{Systematic Uncertainties} \label{sec:sys}
The systematic uncertainties on the \gagg\ limits arise from the 
following sources:
\begin{itemize}
\item Uncertainty on the product 
$\left.Q_L\beta\middle/\left(1+\beta\right)\right.$ in Eq.~\eqref{eq:ps}: 
In order to extract the loaded quality factor $Q_L$ and the coupling 
coefficient $\beta$, a fitting of the measured results of the cavity 
scattering matrix was performed. A relative uncertainty of 5\% is 
assigned to this product, after a comparison of the measurements at 
$T_\text{c}\simeq155$~mK with a prediction extrapolated from the measurements 
at room temperature. More details about the measurements of cavity properties 
can be found in Ref.~\cite{TASEHInstrumentation}.

\item Uncertainty on the noise temperature \ta\ from the RMS of 
the measurements in the calibration: 
$\left. \Delta \ta\middle/\ta\right.= 2.3\%$ 
(see Sec.~\ref{sec:calibration} and Fig.~\ref{fig:hemtcalvsf}).

\item Uncertainty on the noise temperature \ta\ from the largest difference 
between the value determined by the calibration and that from the axion 
data: $\left. \Delta \ta\middle/\ta\right.= 4\%$ 
(see Sec.~\ref{sec:calibration} and Fig.~\ref{fig:hemtcalvsf}). 

\item Uncertainty on the misalignment $\delta f_m$ between the true 
axion frequency $f_a$ and the lower bin boundaries in the merged spectrum 
(see Sec.~\ref{sec:merge}). MORE DESCRIPTION.

\item Uncertainty from the choice of the SG-filter parameters: i.e.  
the window width and the order of the polynomial in the SG filter. At the 
beginning of the data taking, a preliminary optimization was performed: a 
window width of 201 bins and a 4$^\text{th}$ order polynomial were used for 
the first analysis of the CD102 data (see Sec.~\ref{sec:ana}). 
This choice is kept for the central results. 
Nevertheless, various methods of optimization are also explored. The goal 
of the optimization is to find a set of SG-filter parameters that only 
model the noise spectrum and do not remove a real signal. 
The methods include:
\begin{itemize}
 \item Minimize the difference between the two outputs returned by the SG 
filter, when the SG filter is applied to: (i) the real data only, and (ii) 
the sum of the real data and the simulated axion signals. 
 \item Minimize the difference between the output returned by the 
 SG filter and the function ${\cal G}_\text{noise}$ 
that models the noise spectrum (derived by fitting the CD102 data), 
when the SG filter is applied to the sum of the simulated noise based on 
${\cal G}_\text{noise}$ and the simulated axion signals. 
See Fig.~\ref{fig:sgcompare} for an example of the 
simulated spectrum, the function ${\cal G}_\text{noise}$, and the 
output returned by 
 the SG filter when a 3$^\text{rd}$-order polynomial and a window of 141 
 bins are chosen; the differences from all the frequency bins are summed 
 together when performing the optimization.
 Figure~\ref{fig:sgoptimize} shows the difference 
as a function of window widths when the order of polynomial is 
 set to three, four, and six. 
 \item Compare the mean $\mu_\text{noise}$ and the width $\sigma_\text{noise}$ 
of the measured power after applying the SG filter, 
assuming that no signal is present in the 
data. See Fig.~\ref{fig:noisegauss} for an example distribution 
of the measured power from the averaged spectrum of a 
single scan; %, when the cavity resonant frequency 
%is 4.798147~GHz; 
a Gaussian fit is performed to extract 
$\mu_\text{noise}$ and $\sigma_\text{noise}$. Given the nature of the 
thermal noise, the two variables are supposed to be related to 
each other if proper window width and order are chosen:
\begin{equation*} 
\sigma_\text{noise} = \frac{\mu_\text{noise}}{\sqrt{N_\text{spectra}}},
\end{equation*}
where $N_\text{spectra}$ is the number of spectra for averaging and 
is related to the amount of integration time for each frequency step. In 
general, $N_\text{spectra}=1920000-2520000$. 
\end{itemize}

In addition, one could choose to optimize for each frequency step 
individually, optimize for a certain frequency step but apply the results to 
all data, or optimize by adding all the frequency steps together. 
%Figure~\ref{fig:syssgfilter} shows that 
The deviations from the central results using different optimization 
approaches are in general within 1\% and the 
maximum deviation of 1.8\% 
on the \gagg\ limit is used as a conservative estimate of the systematic 
uncertainty from the SG filter. 

\end{itemize}

%The first source of the systematic uncertainty 
%has negligible effect on the limits of \gagg\ while the 
%latter three sources are studied and added in quadrature to obtain the total 
%systematic uncertainty. 
The effects on the \gagg\ limits from these five sources are studied and added in 
quadrature to obtain the total systematic uncertainty. 
The systematic uncertainties on the \gagg\ limits 
are displayed together with the central results in Sec.~\ref{sec:results}. 
%Overall the total relative systematic uncertainty is $\approx 3.4\%$.
Overall the total relative systematic uncertainty is $\approx XXX\%$.

\begin{figure} [htbp]
  \centering
  \includegraphics[width=8.6cm]{figures/GeneratedSpectrum_Optimized_SGFilter_NPar_3_Window_141.pdf}
  \caption{Upper panel: 
 The simulated spectrum (red), including the axion signal and the 
noise, is overlaid with the function that models the noise 
${\cal G}_\text{noise}$ (black) and the 
output returned by the SG filter (green). Lower panel: The ratio of the output 
returned by the SG filter to the function ${\cal G}_\text{noise}$.}
  \label{fig:sgcompare}
\end{figure}


\begin{figure} [htbp]
  \centering
%  \includegraphics[width=0.4\textwidth,height = 0.25\textwidth]{figures/chi2_Different_Order_Window_SGFilter.png}
  \includegraphics[width=8.6cm]{figures/chi2_Different_Order_Window_SGFilter.png}
  \caption{The difference between the output returned by the SG filter 
  and the function that models the noise spectrum, when various values of 
  window widths and 
  a 3$^\text{rd}$, a 4$^\text{th}$, or a 
  6$^\text{th}$-order polynomial are applied in the SG filter. In this 
  figure, the best choice is a 4$^\text{th}$-order polynomial with 
  a window width of 241 data points (bins). }
  \label{fig:sgoptimize}
\end{figure}
 


\begin{figure} [htbp]
  \centering
  \includegraphics[width=8.6cm]{figures/sysSG_temphistogram.png}
  \caption{An example of the distribution of the measured power after 
applying the SG filter, when 
the cavity resonant frequency is 4.798147~GHz. The distribution contains 
1600 entries and each entry corresponds to the measured power 
in one frequency bin, averaged
over 1920000 subspectra. The mean and the width returned by 
a Gaussian fit to the distribution are used to determine the best choice of 
SG parameters. The mean $\mu_\text{noise}=3.2\times10^{-20}$~W in 
a 1-kHz frequency bin would imply a noise temperature of 2.3~K.}
  \label{fig:noisegauss}
\end{figure}
 

%\begin{figure} [htbp]
%  \centering
%  \includegraphics[width=8.6cm]{figures/sys_compareSG_4_201.png}
%  \caption{The ratios of the limits on \gagg\ due to the different choices 
% of the window width and the order of polynomial in the SG filter, with 
%respect to 
% the central results (a window width of 201 bins and the 4$^\text{th}$-order 
% polynomial). The window width of 241 bins and the 4$^\text{th}$-order 
% polynomial are obtained from the optimization after injecting an axion 
%signal on top of a simulated noise spectrum. The window width of 189 bins and 
%the 3$^\text{rd}$-order polynomial are obtained from the optimization 
% after comparing the means and the widths of the measured power distributions.}
%  \label{fig:syssgfilter}
%\end{figure}
 


%\section{Results} \label{sec:results}

Figure~\ref{fig:glimit} shows the limits on the axion-two-photon coupling 
\gagg\ and the ratio of the limits on the dimensionless parameter \ggamma\ 
with respect to the KSVZ benchmark value ($\left|g_\text{KSVZ}\right|=0.97$).  
The blue error band indicates the systematic uncertainties as discussed in 
Sec.~\ref{sec:sys}. No limits are placed for the frequency ranges of 
4.710170 -- 4.710190~GHz and 4.747301 -- 4.747380~GHz, which correspond to 
the external signals during the collection of the CD102 data. The limits on 
\gagg\ range from \lolimit\GeVinv\ to \hilimit\GeVinv, with an average 
value of \avelimit\GeVinv; the lowest value comes from the frequency bins with 
additional eight times more data from the rescans, while the highest value 
comes from the frequency bins near the boundaries of the spectrum. 
Figure~\ref{fig:gaggall} displays the \gagg\ limits obtained by TASEH 
together with those from the previous searches. 
The results of TASEH exclude the models with the axion-two-photon
coupling $\gagg\gtrsim \avelimit\GeVinv$, a factor of ten above the benchmark
KSVZ model for the mass range $\mlo < \ma < \mhi \muevcc$ (corresponding to 
the frequency range of $\flo < f_a < \fhi$~GHz). 


The central results shown in Figs.~\ref{fig:glimit}--\ref{fig:gaggall} are 
obtained assuming an axion signal line shape that follows 
Eq.~\eqref{eq:simplesignal}. The analysis that merges bins without 
assuming a signal line shape %[$L_{k}=\left.1\middle/5\right.$ 
%in Eq.~\eqref{eq:merge_weight}] 
results in $\approx5.5$\% larger values on the 
\gagg\ limits. If a Gaussian signal line shape with an FWHM of 2.5~kHz,  
about half of the axion line width in Eq.~\eqref{eq:simplesignal}, is 
assumed instead, the limits will be $\approx3.8$\% smaller than the central results. 



\begin{figure*} [htbp]
  \centering
  \includegraphics[width=12.9cm]{figures/TASEHonly_limits.png}
%  \includegraphics[width=17.2cm]{figures/TASEHonly_limits.png}
%  \includegraphics[width=8.6cm]{figures/TASEHonly_limits.png}
  \caption{The limits on \gagg\ and the ratio of the limits on 
\ggamma\ relative to $\left|g_\text{KSVZ}\right|=0.97$ 
  (inset) for the frequency range of 
\flo--\fhi~GHz. The blue error band indicates the systematic 
  uncertainties as discussed in Sec.~\ref{sec:sys}. The yellow 
 band in the inset shows the allowed region of \ggamma\ vs. $m_a$ 
 from various QCD axion models, while the blue and red dashed lines are the 
values predicted by the KSVZ and DFSZ benchmark models, respectively}
  \label{fig:glimit}
\end{figure*}


\begin{figure*} [htbp]
  \centering
 \includegraphics[width=12.9cm]{figures/RealData_limit_allexp.png}
  \caption{The limits on the axion-two-photon coupling \gagg\ for the 
frequency ranges of 0--8~GHz, from the CD102 data of TASEH and previous 
searches performed by the ADMX, CAPP, and HAYSTAC Collaborations. The gray 
band indicates the allowed region of \gagg\ vs. $m_a$ from various QCD axion 
models while the blue and red dashed lines are the values predicted by the 
KSVZ and DFSZ benchmark models, respectively.}
  \label{fig:gaggall}
\end{figure*}


%\begin{figure} [htbp]
%  \centering
% \includegraphics[width=8.6cm]{figures/limitratio_weights.png}
%  \caption{The ratios of the limits on \gagg\ from the merging without 
% assuming a signal line shape (blue) and from the merging with a 
% Gaussian weight (orange), with respect to the central results.}
%  \label{fig:limitratio}
%\end{figure}

%\newpage
%\section{Conclusion} \label{sec:conclusion}
This paper presents the first results of a search for axions for the mass 
range $\mlo < \ma < \mhi \muevcc$, using the CD102 data collected by the 
Taiwan Axion Search Experiment with Haloscope from October 13, 2021 
to November 15, 2021. 
Apart from the external signals, no candidates with a significance more than
3.355 were found. The experiment excludes models with the 
axion-two-photon coupling $\gagg\gtrsim \avelimit\GeVinv$ at 95\% C.L.,
 a factor of ten 
above the benchmark KSVZ model. The sensitivity on \gagg\ reached by TASEH 
is three orders of magnitude better than the existing limits. 
It is also the first time that a haloscope-type experiment places 
constraints in this mass region. The synthetic 
axion signals were injected after the collection of data and the 
successful results validate the data acquisition and the analysis procedure. 

The target of TASEH is to search for axions for the mass range of 
16.5--20.7\muevcc\ corresponding to a frequency range of 4--5~GHz, with a 
capability to be extended to 2.5--6~GHz in the future. 
In the coming years, several upgrades are expected, including: the use of a 
quantum-limited Josephson parametric amplifier as the first-stage amplifier, 
the replacement of the existing dilution refrigerator with a new one that has 
a magnetic field of about 9~Tesla and a larger bore size, and the development 
of a new cavity with a significantly larger effective volume. %These upgrades 
%will reduce the added noise by a factor of 10 and increase the magnetic 
%field and the cavity volume by a factor of 1.125 and 5, respectively. 
With the improvements of the experimental setup and several years of data 
taking, TASEH is expected to probe the QCD axion band in the target mass range.




\begin{acknowledgments}

\end{acknowledgments}

%\section{Appendixes}

\bibliography{short}% Produces the bibliography via BibTeX.

\end{document}
%
% ****** End of file apssamp.tex ******
