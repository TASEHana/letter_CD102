%\newpage
%\section{Conclusion} \label{sec:conclusion}
This paper presents the first results of a search for axions for the mass 
range $\mlo < \ma < \mhi \muevcc$, using the CD102 data collected by the 
Taiwan Axion Search Experiment with Haloscope from October 13, 2021 
to November 15, 2021. 
Apart from the external signals, no candidates with a significance more than
3.355 were found. The experiment excludes models with the 
axion-two-photon coupling $\gagg\gtrsim \avelimit\GeVinv$ at 95\% C.L.,
 a factor of ten 
above the benchmark KSVZ model. The sensitivity on \gagg\ reached by TASEH 
is three orders of magnitude better than the existing limits. 
It is also the first time that a haloscope-type experiment places 
constraints in this mass region. The synthetic 
axion signals were injected after the collection of data and the 
successful results validate the data acquisition and the analysis procedure. 

The target of TASEH is to search for axions for the mass range of 
16.5--20.7\muevcc\ corresponding to a frequency range of 4--5~GHz, with a 
capability to be extended to 2.5--6~GHz in the future. 
In the coming years, several upgrades are expected, including: the use of a 
quantum-limited Josephson parametric amplifier as the first-stage amplifier, 
the replacement of the existing dilution refrigerator with a new one that has 
a magnetic field of about 9~Tesla and a larger bore size, and the development 
of a new cavity with a significantly larger effective volume. %These upgrades 
%will reduce the added noise by a factor of 10 and increase the magnetic 
%field and the cavity volume by a factor of 1.125 and 5, respectively. 
With the improvements of the experimental setup and several years of data 
taking, TASEH is expected to probe the QCD axion band in the target mass range.


